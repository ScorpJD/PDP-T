\documentclass[10pt,twoside]{article}
\usepackage[utf8]{inputenc}
\usepackage{natbib}
\usepackage{graphicx}
\usepackage{booktabs}
\usepackage{amsmath}
\usepackage{amsfonts}
\usepackage{adjustbox}
\usepackage[table,xcdraw]{xcolor}
\usepackage{amsmath}
\usepackage[]{todonotes}
\usepackage{fancyhdr}
\usepackage{multirow}

% Idioma
%\usepackage[spanish, es-tabla]{babel}

\usepackage[Alg.]{algorithm}
\usepackage{algpseudocode}

\newcommand{\ignore}[1]{}


\title{Pickup and Delivery Problem with Transfers}
\author{\emph{Santiago Hincapie, Catalina Lesmes} and \emph{Juan Carlos Rivera}\\
\vspace{0.3cm}
\small{\tt{shinca12@eafit.edu.co, clesmes@eafit.edu.co, jrivera6@eafit.edu.co}}\\
Department of Mathematical Sciences\\
School of Science\\
EAFIT University\\
Medellín -- Colombia}
\date{}

\usepackage{anysize}
\marginsize{3cm}{2cm}{2cm}{3cm}

% Configurar encabezado y pie de página
\pagestyle{fancy}
\lfoot[\date{August 7, 2017}]{\date{August 7, 2017}}
\rfoot[\thepage]{\thepage}
\cfoot[]{}
\rhead[Lesmes, Potes and Rivera]{Pickup and Delivery Problem with Transfers}
\lhead[]{}

\sloppy

\begin{document}

\maketitle

\begin{abstract}
In this article we will be talking about Pickup and Delivery Problems with
Transfers (PDP-T), the main idea is to show first the Pickup and Delivery
Problem (PDP), which is the problem from which the PDP-T derives from, later
show a description of the problem and the mathematical formulation to get a
clear view of what we are going to solve, which leads us to present different
ways of solving this kind of problems.


\end{abstract}

\section{Introduction}\label{sec_intro}

In many logistics problems we require to pick some products in one place and
take them to another place, which is exactly why we define the pickup and
delivery problem (PDP), in this problem a set of vehicles pick up and deliver
a set of items. The goal is to deliver as many items as possible at the lowest
cost while obeying a set of constraints, such as time windows and capacities.
The PDP is a well-studied, NP-hard problem, so approximation algorithms and
heuristics have been developed to address variants of the PDP.
There are many techniques we can use to solve the PDP problem, we can find
genetic algorithms, various metaheuristics, taboo search heuristics and a
branch and cut algorithm.

To make this logistic problem bigger we can make the vehicle that is delivering
the product transfer it to another vehicle before the delivery its done. By
defining that we have the PDP with Transfers (PDP-T), in which we consider
transferring items between vehicles. We can convert the PDP problem into a
PDP-T problem by adding some variables and constraints.



\section{Literature Review}\label{sec_SoA}

As we saw before, there are different methods to solve the PDP-T problem.

We can use a branch and cut method using Benders Decomposition. In this method,
the set of constraints is decomposed into pure integer and mixed constraints,
and then a branch-and-cut procedure is applied to the resulting pure integer
problem, by using real variables and constraints related as cut generators.
The key on the success of this method is that those constraints defined by a
logical sentence are not modeled using the big-M technique, as usual in a
branch-and-bound methodology behind the original PDP formulation. This method
may be applied only when the objective function is either pure real or pure
integer. \citep{Cortes2010}

Another method to solve it is Very Large Neighborhood Search with Transfers
(VLNS-T) is based on the Adaptive Very Large Neighborhood Search (VLNS).
The VLNS algorithm  uses  simulated  annealing  to  randomly  choose
neighboring schedules and iteratively improve the schedule. Neighboring
schedules are formed by removing random items and reinserting them with
heuristics. So based on the VLNS algorithm for the PDP without transfers, a
variant of simulated annealing in which the neighborhood of states is very
large. In this case we remove random items from the schedule and then reinsert
them with multiple heuristics to find neighbors. \citep{Cata}

There are also have some special cases of the PDP-T problem, as the Pickup and
Delivery Problem with Shuttle Routes (PDP-S) which relies on a structured
network with two categories of routes. Pickup routes visit a set of pickup
points independently of their delivery points and end at one delivery point.
Shuttle routes are direct trips between two delivery points. Requests can be
transported in one leg (pickup route) or two legs (pickup route plus shuttle
route) to their delivery point. The PDP-S applies to transportation systems
with a multitude of pickup points and a few, common delivery points.
\citep{Masson2010}


\section{Problem Description and Mathematic Formulation}\label{sec_mathmod}
In the PDP, a heterogeneous vehicle fleet based at multiple terminals must
satisfy a set of transportation requests. Each request is defined by a pickup
point, a corresponding delivery point, and a demand to be transported between
these locations\citep{Parragh2008}, now, PDP-T allow the option for passengers
to transfer between vehicles, provided that the locations of the transfer
points are fixed and known.\\

\subsection{Problem Formulation}
Let $N$ be the set of transportation requests. For each transportation request
$i\in N$ a load of size $\bar{q_i}\in \mathbb{N}$ has to be transported from a
set of origins $N^+_i$ to a set of destinations $N_i^+$. \\
Each load is subdivided as follows
$\displaystyle\bar{q_i} = \sum_{j\in N_i^+}q_j = - \sum_{j\in N_i^-} q_j$,
i.e., positive quantities for pickups and negative quantities for deliveries.\\
Define $\displaystyle N^+ := \cup_{i\in N} N_i^+$ as the set of all origins and
$\displaystyle N^- := \cup_{i\in N} N_i^-$ as the set of all destinations.
Let $V := N^+\cup N^-$.\\
Furthermore, let $M$ be the set of vehicles. Each vehicle $k\in M$ has a
capacity $Q_k \in \mathbb{N}$ a start location $k^+$ and an end location $k^-$.
\\Define $M^+ := \{k^+|k\in M\}$ as the set of start locations and
$M^- := \{k^-|k\in M\}$ as the set of end locations. Let $W := M^+ \cup M^-$.\\
For all $i, j \in V \cup  W$ let $d_{ij}$ denote the travel distance, $t_{ij}$
the travel time and $c_{ij}$ the travel cost. Note that the dwell time at
origins and destinations can be easily incorporated in the travel time and
therefore will not be considered explicitly.\\

To formulate the PDP as a mathematical program we introduce four types of
variables:
for $i\in N$ and $k\in M$:
$$
z_i^k =
\begin{cases} 1 &  \text{if transportation request } i
\text{is assigned to vehicle } k \\
0 & Otherwise  \end{cases}
$$
for $(i, j)\in (V \times V)\cup \{(k^+,j)|j\in V\} \cup \{(j,k^-)|j\in V\}$
and $k\in M$:
$$
x_{ij}^k = \begin{cases} 1 & \text{if vehicle } k  \text{travels from location }
 i \text{to location } j \\ 0 & Otherwise  \end{cases}
$$
$D_i$  with $(i \in  V\cup W)$, specifying the departure time at vertex $i$ and
\\$y_i$ with $(i \in  V\cup W)$, specifying the load of the vehicle arriving at
vertex $i$
\citep{Parragh2008}.\\
\newpage
All this information are summarize in the next table:
\begin{center}
\begin{tabular}{ll}
  \hline
  Object   & Meaning                                             \\
  \hline
  $M$      & Set of vehicles                                     \\
  $C$      & Set of requests                                     \\
  $T$      & Set of transference point                           \\
  $M^+$    & Set of origin depots for vehicles                   \\
  $M^-$    & Set of destination depots for vehicles              \\
  $N^+$    & Set of origin nodes for requests                    \\
  $N^-$    & Set of destination nodes for requests               \\
  $N$      & Set of nodes associated with requests               \\
  $V$      & Set of nodes                                        \\
  $q_{ij}$ & Size of request $i \in C$                           \\
  $Q_k$    & Capacity of vehicle $k \in K$                       \\
  $t_{ij}$ & Minimum ride time from node $i$ to node $j$         \\
  $c_{ij}$ & The travel cost                                     \\
  $d_{ij}$ & The travel distance                                 \\
  $z_i^k$  & bind transportation request and vehicles            \\
  $x_{ij}$ & bind routes and vehicles information                \\
  $D_i$    & specifying the departure time at specific vertex    \\
  $y_i$    & specifying the load of the vehicle arriving         \\
  \hline
\end{tabular}
\end{center}
\citep{Parragh2008} \\

\vspace{1cm}
Now the problem is

\begin{align}
& {\text{minimize}}
& & \sum_{i,j\in V \cup W} dij && \\
& \text{subject to}
& & \sum_{k\in M} z_i^k = 1 & \text{for all }& i\in N \\
&&& \sum_{j\in V\cup W} x_{il}^k = z_{i}^k & \text{for all } &
    i \in N,l\in N^+_i\cup N_i^- k\in M\\
&&& \sum_{j\in V\cup\{k^-\}} x^k_{k^+ j} = 1 & \text{for all }& k\in M\\
&&& \sum_{j\in V\cup\{k^+\}} x^k_{k^- j} = 1 & \text{for all }& k\in M\\
&&& D_{k^+} = 0 & \text{for all }& k\in M\\
&&& D_p \leq D_q & \text{for all }& i\in N, p\in N_i^+ q\in N_i^-\\
&&& x_{ij}^k = 1\Rightarrow D_i + t_{ij} \leq D_j & \text{for all } &
    i,j \in V\cup W, k\in M\\
&&& y_{k^+} = 0 & \text{for all }& k\in M\\
&&& y_l \leq \sum_{k\in M} Q_kz_i^k & \text{for all } &
    i\in N, l\in N_i^+\cup N_i^-\\
&&& D_i \geq 0 & \text{for all }& i\in V\cup W\\
&&& y_i \geq 0 & \text{for all }& i\in V\cup W
\end{align}

Constraint (2) ensures that each transportation request is assigned to exactly
one vehicle. By constraint (3) a vehicle only enters or leaves a location $l$
if it is an origin or a destination of a transportation request assigned to that
vehicle. The next (4) and (5) make sure that each vehicle starts and ends at
the correct place. Also the constraints number (6), (7), (8) and (11) together
form the precedence constraints the others together form the capacity
constraints.  Constraints (9), (10) and (12) together form the capacity
constraints.\\
This mathematical model, model the PDP problem, now for a PDP-T model in the
literature introduce the transfer point idea, and the idea for the extended
model is to iterative add constrains that involves this transfer points.

\section{Computational Experimentation}\label{sec_results}
In the next section we talk about the computational method that solve the PDP-T
problem describe in the past. In this section present two approach base on
\cite{multistart} and \cite{CITAR}, then we use GRASP meta heuristic describe in
\cite{GRASP} and \cite{bGRASP}, to improve the solution also, use stochastic

For the computational result we use python 3.5.3 running on Intel Xeon E5-1620V4
, 64-bit processor with 12 gigabyte RAM and Debian 9. We used problem instances
from those in for the PDP, which are related to the well-known Solomon instances.
The datasets are available at http://www.sintef.no/Projectweb/TOP/PDPTW/ , for
each instance we introduce a random number of transfer points, in a random node,
to become the original PDP instance in a PDP-T instance.\\

Each data set contains $X - Y$ hence, the lengths of the arcs in the network are
$L_2$ distances. For the locations of the vehicle depots, we randomly generated
origin and final depots for each vehicle, scattered over the region formed by
the nodes. The number of vehicles are also random generated, and finally we
ignore the time windows restriction because are not consider in the model.

\subsection{Solution methods}\label{sec_sol}
In general we first solve the PDP problem using different method (that would be
describe in this section) and then we add the transfership option to the PDP
solution, so transefership option always increase the computational time of the
particular instance.

\subsubsection{Greedy approach base on \cite{multiagent}}
We implement the greedy approach describe at \cite{multiagent} that basically,
iterate through every item and vehicle and insert the best pickup and delivery
action into the schedule, always choose the option that increases the cost the
least, we repeat the process until no unassigned items remain. Then we add the
transfer option with another greedy idea, inspired on clarke saving algorithm,
this method compute de saving of the transfer, and if improve the solution,
modify the solution adding this new note.

\subsubsection{Multistart base on \cite{multistart}}
As descrive at \cite{multistart}, this approach is merge many ideas.


\subsection{GRASP}
We implement a functional GRASP meta heuristica describe at \cite{bGRASP}, using
the some python tricks, we create a general method that takes other method as
parameter and then performce this meta heuristic.

\newpage
\subsection{Computational results}\label{sec_eval}

\begin{table}[H]
\caption{Greedy approach}
\begin{tabular}{lllll}
  \hline
  Instance & Optimal Value & Time   & Optimal Value (with transfers) & Time (transfer) \\
  \hline
  10n2a    & 1202.26       & 35.29  & 1160.01                        & 24.46  \\
  10n4a    & 2205.11       & 128.79 & 2205.11                        & 710.34 \\
  10n8a    & 954.52        & 226.63 & 954.52                         & 702.35 \\
  \hline
\end{tabular}
\end{table}

\begin{table}[H]
\caption{Greedy approach (GRASP)}
\begin{tabular}{lllll}
  \hline
  Instance & Optimal Value & Time   & Optimal Value (with transfers) & Time (transfer) \\
  \hline
  10n2a    & 1183.26       & 46.23  & 1134.21                        & 60.61  \\
  10n4a    & 2143.13       & 339.91 & 2003.43                        & 530.34 \\
  10n8a    & 842.43        & 527.54 & 842.43                         & 1241.76\\
  \hline
\end{tabular}
\end{table}

\begin{table}[H]
\caption{MULTISTART}
\begin{tabular}{lllll}
  \hline
  Instance & Optimal Value & Time   & Optimal Value (with transfers) & Time (transfer) \\
  \hline
  10n2a    & 965.26        & 82.29  & 937.01                         & 54.46  \\
  10n4a    & 1205.11       & 387.91 & 1205.11                        & 921.44 \\
  10n8a    & 754.52        & 563.12 & 754.52                         & 1002.51\\
  \hline
\end{tabular}
\end{table}

\begin{table}[H]
\caption{MULTISTART (GRASP)}
\begin{tabular}{lllll}
  \hline
  Instance & Optimal Value & Time   & Optimal Value (with transfers) & Time (transfer) \\
  \hline
  10n2a    & 935.26        & 512.36 & 933.21                         & 60.61  \\
  10n4a    & 1205.11       & 339.91 & 1205.11                        & 530.34 \\
  10n8a    & 742.43        & 1026.41& 741.32                         & 1412.98\\
  \hline
\end{tabular}
\end{table}

\section{Conclusions}\label{sec_conclusions}
\begin{itemize}
\item As we can see in the computational results, in general GRASP meta heuristic
  improve the solution, but the gap between the GRASP solution and the Non-GRASP
  solution are not significant, in the other hand, the CPU time gap's are so big,
  so in general we cant recomend use this meta heuristic for this problem, this
  can be a result of the implementation, as the GRASP implementation is very
  general does not takes spetial information of the original methods, and may
  recompute some details of the original method.

\item In general MULTISTART has a better results than the Greedy approach, but
  also are slower. Using the computational results, we can conclude that the
  MULTISTART method is better than the Greedy.
\end{itemize}

\newpage
{\small
\bibliographystyle{authordate1}
\bibliography{bibliografia}{\nocite{*}}}


\end{document}

